%============= macro start =============%
\documentclass[a4paper,10pt,oneside,openany]{jsbook}

%% packages %% 
\usepackage{amsmath,amssymb} 		% ams仕様の数学記号を用いる
\usepackage{bm}				   		% ベクトルの記述に用いる
\usepackage[dvipdfmx]{graphicx}		% 画像の表示
\usepackage{titlesec}					% 章節などの大きさを変更
\usepackage[usenames]{color}			% 色の変更
\usepackage{listings,jlisting} 			% ソースコードを記述する際に使う
\usepackage{ulem}						% いろいろな下線を書く
%\usepackage{ascmac}				% 複数行にわたる文章を囲む
%\usepackage{makeidx}				% 索引の作成
%\usepackage{picture}				% 作画用

%% マージンの設定 %%
\setlength{\textwidth}{\fullwidth}
\setlength{\textheight}{44\baselineskip}
\addtolength{\textheight}{\topskip}
\setlength{\voffset}{-0.6in}

%% 数式番号の表示に関わる設定 %%
\makeatletter
  \renewcommand{\theequation}{\arabic{chapter}.\arabic{section}.\arabic{equation}}
  \@addtoreset{equation}{section}
\makeatother

%% ソースコード表示のデザインに関わる設定 %%
\lstset{
  language={C},
  basicstyle={\small},
  identifierstyle={\small},
  commentstyle={\small\itshape},
  keywordstyle={\small\bfseries},
  ndkeywordstyle={\small},
  stringstyle={\small\ttfamily},
  frame={tb},
  breaklines=true,
  columns=[l]{fullflexible},
  numbers=left,
  xrightmargin=0zw,
  xleftmargin=3zw,
  numberstyle={\scriptsize},
  stepnumber=1,
  numbersep=1zw,
  lineskip=-0.5ex
}
%% 色の設定 %%                                                                                                                                                                                                                                                                                                                                                               
\definecolor{teal}{RGB}{0,128,128}
\definecolor{powderblue}{RGB}{176,224,230}
\definecolor{darkslateblue}{RGB}{72,61,139}
\definecolor{darkslategray}{RGB}{47,79,79}
\definecolor{lightcyan}{RGB}{224,255,255}

%% ヘッダ・章・節のデザインに関わる設定 %%
% chapter                                                                                                                                                                                                                                                                                                                                                              
%\titleformat{\chapter}[block]
%{}{}{0pt}{
%  \fontsize{40pt}{40pt}\selectfont\filleft
%}[
%  \hrule \Large{\filleft 第 \thechapter 章}
%]
% section                                                                                                                                                                                                                                                                                                                                                              
\titleformat{\section}[block]
{}{}{0pt}
{
  \colorbox{teal}{\begin{picture}(0,15)\end{picture}}
  \hspace{0pt}
  \normalfont \LARGE\bfseries \thesection
  \hspace{0pt}
}
[
\begin{picture}(100,0)
  \put(3.3,23){\color{teal}\line(1,0){400}}
\end{picture}
\\
\vspace{-50pt}
]
% subsection                                                                                                                                                                                                                                                                                                                                                           
\titleformat{\subsection}[block]
{}{}{0pt}
{
  \colorbox{darkslateblue}{\begin{picture}(0,10)\end{picture}}
  \hspace{0pt}
  \normalfont \large\bfseries \thesubsection
  \hspace{0pt}
}
[
\begin{picture}(100,0)
  \put(3.4,15.2){\color{darkslateblue}\line(1,0){300}}
\end{picture}
\\
\vspace{-30pt}
]

%% フロントページ %%
\title{{\Huge \textbf{数値計算/シミュレーションハンドブック}}\\ {\small Ver. 1.0.0}}
\author{勉強会参加メンバー}
\date{\today}

%% 索引の作成 %%
%\makeindex

%% 記号の定義 %%
\DeclareMathOperator{\divergence}{\mathrm{div}\,}  %texでは\divは割算の記号
\DeclareMathOperator{\grad}{\mathrm{grad}\,}
\DeclareMathOperator{\rot}{\mathrm{rot}\,}

%============= macro end =============%


\begin{document}

\maketitle
\frontmatter
\tableofcontents
\mainmatter

{\LARGE 記号のリスト}

\begin{table}[htb]
  \begin{tabular}{ll}
    $d$ & 次元 \\
    $h_t$ & 時間のステップ幅 \\
    $h_x$ & 空間のステップ幅 \\
  \end{tabular}
\end{table}
\chapter{本稿に関して}

\section{本稿の目的}
しばしば耳にする「数値計算」や「数値シミュレーション」,「仮想実験」などは一度は少し踏み込んだところまで触っておきたいもの.
そのような簡単な理由から,分野問わず幅広い知見から上記項目をまとめていくことを目的にしている.
特に,専門的になりすぎず初学者がとっつきやすいノート(ハンドブックと名付けた理由)になればと思っている.
ネット上にはたくさんの知識が転がっている.
特に,同じような目的からかwikiなりまとめなりのサイトも増え始めている.
これらネットに転がっている知識もこのノートに集約できるなら最高の幸せでる.
いつの日かこのノートのおかげで数値計算に入り込むことができたという人が現れてくれたら嬉しいな.
なお,本稿のデザインは及びから拝借している.
この場を持って感謝申し上げる.

\section{ルール}
短いコードならpdfに書き込み可.



\section{\TeX}

索引はindex\index{インデックス}で可能.


\section{事前準備}


\chapter{線形方程式}

\section{線形方程式の解法の種類}
\begin{itemize}
\item[1] 直接法
	\begin{itemize}
	\item[i] Gaussの消去法
	\item[ii] LU分解
	\end{itemize}
\item[2] 反復法
	\begin{itemize}
	\item 定常法
		\begin{itemize}	
		\item[i] SOR法 (Successive Over-Relaxation: 逐次加速緩和法)
		\item[ii] Gauss-Seidel法
		\item[iii] Jacobi法
		\end{itemize}
	\item 非定常法
		\begin{itemize}
		\item[i] Krykov部分空間法
		\item[ii] CG (Conjugate Gradient: 共役勾配法)
		\item[iii] BiCGSTAB (Bi-Conjugate Gradient Stabilized)
		\item[iv] GMRES (Generalized Minimal Residual)
		\end{itemize}
	\end{itemize}
\end{itemize}




\chapter{常微分方程式の数値計算}
\begin{abstract}
偏微分方程式の数値計算において
\end{abstract}

\section{問題設定}
常微分方程式
$$
\frac{du(t)}{dt}=f(t,u(t))\quad(t>0)
$$
について考えよう.

\subsection{Runge-Kutta法}

\chapter{偏微分方程式の差分法}
%% abstract %%
\begin{abstract}
もっとも基礎的な数値スキームである有限差分法(Finite Difference Method; FDM)を用いて,いくつかの有名な線形偏微分方程式を計算する.
\end{abstract}


%% Poisson %%
\section{Poisson方程式}
線形のPoisson方程式は
\begin{equation}
\label{e:poisson}
-\Delta u(x)=f(x)\quad(x\in\Omega)
\end{equation}
と記述される.

%
%\subsection{Helmholtz方程式}
%Helmholtz方程式
%\begin{equation}
%\label{e:helmholtz}
%-\Delta u(x)=\lambda u(x)\quad(x\in\Omega)
%\end{equation}



%% heat %%
\section{熱方程式}
線形の熱方程式は
\begin{equation}
\label{e:heat}
\partial_t u(t,x)-c\Delta u(t,x)=f(t,x)\quad(t>0,\ x\in\Omega)
\end{equation}
と記述される.ここで,$\Omega$は$\mathbb{R}^d$の単連結な領域であり,$c>0$は拡散係数(または粘性係数)と呼ばれる(今回は)定数である.


\subsection{ 1次元の問題}
ここでは,$\Omega=(0,1)$とし
\begin{equation}
u(0,x)=g(x)\quad(0\le x\le1)
\end{equation}
及びDirichlet境界条件
\begin{equation}
u(t,0)=a(t),\quad u(t,1)=b(t)\quad(t\ge0)
\end{equation}
を仮定する.
ここで,$g:[0,1]\to\mathbb{R}$及び$a,b:[0,\infty)\to\mathbb{R}$は与えられた関数で,両立条件として$g(0)=a(0)$と$g(1)=b(0)$を満たすと仮定する.

\subsubsection{陽解法}
時間分割数を$N_t$で空間分割数を$N_x$とする.

内部の方程式を,空間に関して\uwave{2階中心差分},時間に関して\uwave{前進差分}による近似を行うと
\begin{equation}
\label{e:heat1d}
\frac{u^{n+1}_{i}-u^{n}_{i}}{h_t}-c\frac{u^{n}_{i+1}-2u^{n}_{i}+u^{n}_{i-1}}{h_x^{2}}=f_i^n
\end{equation}
を得る.ただし,$u_i^n:=u(nh_t,ih_x)$かつ$f_i^n:=f(nh_t,ih_x)$としている.(離散)方程式\eqref{e:heat1d}は
$$
u^{n+1}_{i}=\lambda u^{n}_{i+1}+(1-2\lambda)u^{n}_{i}+\lambda u^{n}_{i-1}
$$
と書き換えられる.ここで,$\lambda:=ch_t/h_x^{2}$である.


\subsubsection{陰解法}


\subsection{2次元の問題}



\subsection{問題}
すぐに思い浮かぶ問題としては例えば次がある.
\begin{itemize}
\item Neumann境界条件を課した場合はどうなるか.
\item 拡散係数が定数でない場合,すなわち
$$
u_t(t,x)-\divergence(c(t,x)\nabla u(t,x))=f(t,x)\quad(t>0,\ x\in\Omega)
$$
の場合はどうなるか.
\end{itemize}







%% wave %%
\section{波動方程式}
線形の波動方程式は
\begin{equation}
\label{e:wave}
u_{tt}u(t,x)-c^2\Delta u(t,x)=f(t,x)\quad(t>0,\ x\in\Omega)
\end{equation}
と記述される.


\subsection{ 1次元の問題}
初期条件として
\begin{equation}
\label{e:waveinitial}
u(0,x)=g_1(x),\quad u_t(0,x)=g_2(x)\quad(0\le x\le 1)
\end{equation}
及びDirichlet境界条件
\begin{equation}
\label{e:wave1ddirichlet}
u(t,0)=a(t),\quad u(t,1)=b(t)\quad(t\ge0)
\end{equation}
を仮定する.

\subsubsection{陽解法}
内部の方程式は,
$$
\frac{u^{n+1}_{i}-2u^{n}_{i}+u^{n-1}_{i}}{(\delta t)^{2}}-c^{2}\frac{u^{n}_{i+1}-2u^{n}_{i}+u^{n}_{i-1}}{(\delta x)^{2}}=f
$$
すなわち,
$$
u^{n+1}_{i}=\lambda u^{n}_{i+1}+2(1-\lambda)u^{n}_{i}+\lambda u^{n}_{i-1}-u^{n-1}_{i}
$$
と分解される.ここで,$\lambda:=(c\delta t/\delta x)^{2}$.

初期条件は,$u^{0}_{i}=\Phi_{i}$及び
$$
u(\delta t,x)\simeq u(0,x)+\delta t\partial_{t}u(0,x)+\frac{(\delta t)^{2}}{2}\partial_{tt}u(0,x)
$$
より
$$
u^{1}_{i}=\Phi_{i}+\Psi_{i}\delta t+\frac{\lambda}{2}(\Phi_{i+1}-2\Phi_{i}+\Phi_{i-1})
$$
で決まる.ここで,$\Phi_{i}:=\phi(x_i)$及び$\Psi_{i}:=\psi(x_i)$である.なお,境界条件は$u^{n}_{0}=u^{n}_{N}=0$で決まる.

\subsection{2次元の問題}
正方領域$\Omega=(0,1)\times(0,1)$上の波動方程式\eqref{e:wave}-\eqref{e:waveinitial}に対するDirichlet境界値問題(固定境界値問題)
\begin{equation}
\label{e:wavedirichlet}
u(t,x)=a(t,x)\quad(t\ge0,\ x\in\partial\Omega)
\end{equation}
を考える.


\subsubsection{陽解法}


内部の方程式は
$$
\frac{u^{n+1}_{i,j}-2u^{n}_{i,j}+u^{n-1}_{i,j}}{(h_{t})^{2}}-c^{2}\left(\frac{u^{n}_{i+1,j}-2u^{n}_{i,j}+u^{n}_{i-1,j}}{(h_x)^{2}}+\frac{u^{n}_{i,j+1}-2u^{n}_{i,j}+u^{n}_{i,j-1}}{(h_y)^{2}}\right)=0
$$
すなわち,
$$
f
$$

\subsection{問題}
すぐに思い浮かぶ問題としては例えば次がある.
\begin{itemize}
\item Neumann境界条件を課した場合はどうなるか.
\item 拡散係数が定数でない場合,すなわち
$$
u_{tt}(t,x)-\divergence(c(t,x)\nabla u(t,x))=f(t,x)\quad(t>0,\ x\in\Omega)
$$
の場合はどうなるか.
\end{itemize}


%% advection %%
\section{移流方程式}
線形の(スカラー)移流方程式\footnote{advection equation}は
\begin{equation}
\label{e:advection}
\partial_t u(t,x)+\nu\cdot\nabla u(t,x)=f(t,x)\quad(t>0,\ x\in\Omega)
\end{equation}
によって記述される.
ここで,$u:\Omega\to\mathbb{R}$は未知関数,$\nu\in\mathbb{R}^d$はゼロでない定数ベクトルであり,$f=f(t,x):(0,\infty)\times\Omega\to\mathbb{R}$は既知の関数ある.もし$\Omega=\mathbb{R}^d$ならば,解は
$$
u(t,x)=u(0,x-t\nu)+\int_0^t f(s,x+(s-t)\nu)ds\quad(x\in\mathbb{R}^d,t\ge0)
$$
とかける(\cite[Section 2.1.2]{evans})

\subsection{離散化}




\subsection{移流拡散方程式}
前々節と合わせることで,線形の移流拡散方程式\footnote{advection diffusion equation}
\begin{equation}
\label{e:advectiondiffusion}
u_t(t,x)-c\Delta u(t,x)+\nu\cdot\nabla u(t,x)=f(t,x)\quad(t>0,\ x\in\Omega)
\end{equation}
の計算は可能である.

\subsection{問題}
すぐに思い浮かぶ問題としては例えば次がある.
\begin{itemize}
\item Neumann境界条件を課した場合はどうなるか.
\item \eqref{e:advection}において移流係数が定数でない場合\footnote{Liouville方程式と呼ばれることがあるようだ.},すなわち
$$
u_t(t,x)+\nabla(\nu(t,x) u(t,x))=f(t,x)\quad(t>0,\ x\in\Omega)
$$
の場合はどうなるか.ただし,$\nu$は既知のスカラー値関数である.
\item \eqref{e:advectiondiffusion}において移流係数または/及び拡散係数が定数でない場合\footnote{Fokker-Plank方程式と呼ばれる.},すなわち
$$
u_t(t,x)-\divergence(c(t,x)\nabla u(t,x))+\nabla(\nu(t,x)u(t,x))=f(t,x)\quad(t>0,\ x\in\Omega)
$$
の場合はどうなるか.ただし,$\nu$は既知のスカラー値関数である.
\end{itemize}




%%%%%%%%%%%%%%%%%%%%%%%%%%%%%%%%%%%%
\begin{thebibliography}{99}
  \bibitem{evans} L.~C.~Evans, \emph{Partial Differential Equations}, Partial Differential Equations, in the series Graduate studies in mathematics,
Second Edition v. 19, American Math. Society, 2010.
  \end{thebibliography}
\chapter{偏微分方程式の有限要素法}
\chapter{その他の偏微分方程式}
\section{Hamilton-Jacobi方程式}
\section{自由境界問題}
\chapter{モンテカルロシミュレーション}
\chapter{セルオートマトン}
%\printindex			%% index %%

\end{document}
