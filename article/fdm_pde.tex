\chapter{偏微分方程式の差分法}
\begin{abstract}
もっとも基礎的な数値スキームである有限差分法(Finite Difference Method; FDM)を用いて,いくつかの有名な偏微分方程式を計算する.
\end{abstract}

\section{移流方程式}
線形の(スカラー)移流方程式\footnote{[English?]}は
\begin{equation}
\partial_t u+\nu\cdot\nabla u=f\quad\text{in $(0,\infty)\times\Omega$}
\end{equation}
によって記述される.
ここで,$u:\Omega\to\mathbb{R}$は未知関数,$\nu\in\mathbb{R}^d$はゼロでない定数ベクトルであり,$f=f(t,x):(0,\infty)\times\Omega\to\mathbb{R}$は既知の関数ある.もし$\Omega=\mathbb{R}^d$ならば,解は
$$
u(t,x)=u(0,x-t\nu)+\int_0^t f(s,x+(s-t)\nu)ds\quad(x\in\mathbb{R}^d,t\ge0)
$$
とかける(\cite[Section 2.1.2]{evans})

\subsection{離散化}


\section{Poisson方程式}





\section{熱方程式}
線形熱方程式の境界値問題は


以下の記号を用いる.
\begin{itemize}
\item[-] $\delta \ast$ ($\ast=t,x,y$):時間及び空間刻み幅
\item[-] $(t_n ,x_i ,y_j):=(n \delta t,i \delta x,j \delta y)$, ($n,i,j=0,1,2,\cdots$):時空間の格子点
\item[-] $u^{n}_{i,j}:=u(t_n, x_i, y_j)$:時空間の格子点上での未知数の値
\end{itemize}

\subsection{ 1次元の問題}
1次元の線形熱方程式に対する周期境界値問題を考える:
\begin{equation}
\label{e:heat1d}
	\begin{cases}
		u_t(t,x)-cu_{xx}(t,x)=f(t,x)\quad&(t>0,\ 0<x<1)\\
		u(0,x)=g(x)\quad&(0\le x\le 1)\\
		u(t,0)=u(t,1)\quad&(t\ge0).
	\end{cases}
\end{equation}

\subsubsection{陽解法}
内部の方程式は,
$$
\frac{u^{n+1}_{i}-u^{n}_{i}}{\delta t}-c\frac{u^{n}_{i+1}-2u^{n}_{i}+u^{n}_{i-1}}{(\delta x)^{2}}=0
$$
すなわち,
$$
u^{n+1}_{i}=\lambda u^{n}_{i+1}+(1-2\lambda)u^{n}_{i}+\lambda u^{n}_{i-1}
$$
と分解される.ここで,$\lambda:=c\delta t/(\delta x)^{2}$.


\subsection{ 2次元の問題}



\section{波動方程式}
以下の記号を用いる.
\begin{itemize}
\item[-] $\delta \ast$ ($\ast=t,x,y$):時間及び空間刻み幅
\item[-] $(t_n ,x_i ,y_j):=(n \delta t,i \delta x,j \delta y)$, ($n,i,j=0,1,2,\cdots$):時空間の格子点
\item[-] $u^{n}_{i,j}:=u(t_n, x_i, y_j)$:時空間の格子点上での未知数の値
\item[-]  
\end{itemize}

\subsection{ 1次元の問題}
次の波動方程式に対するDirichlet境界値問題(固定境界値問題)を数値計算することを考える.
\begin{equation}
\label{e:wave1d}
	\begin{cases}
		\partial_{tt}u(t,x)-c^{2}\partial_{xx}u(t,x)=0\quad&\text{for $(t,x)\in(0,\infty)\times(0,1)$,}\\
		u(0,x)=\phi(x),\quad \partial_{t}u(0,x)=\psi(x)\quad&\text{for $x\in[0,1]$,}\\
		u(t,0)=u(t,1)=0\quad&\text{for $t\in[0,\infty)$.}
	\end{cases}
\end{equation}

\subsubsection{陽解法}
内部の方程式は,
$$
\frac{u^{n+1}_{i}-2u^{n}_{i}+u^{n-1}_{i}}{(\delta t)^{2}}-c^{2}\frac{u^{n}_{i+1}-2u^{n}_{i}+u^{n}_{i-1}}{(\delta x)^{2}}=0
$$
すなわち,
$$
u^{n+1}_{i}=\lambda u^{n}_{i+1}+2(1-\lambda)u^{n}_{i}+\lambda u^{n}_{i-1}-u^{n-1}_{i}
$$
と分解される.ここで,$\lambda:=(c\delta t/\delta x)^{2}$.

初期条件は,$u^{0}_{i}=\Phi_{i}$及び
$$
u(\delta t,x)\simeq u(0,x)+\delta t\partial_{t}u(0,x)+\frac{(\delta t)^{2}}{2}\partial_{tt}u(0,x)
$$
より
$$
u^{1}_{i}=\Phi_{i}+\Psi_{i}\delta t+\frac{\lambda}{2}(\Phi_{i+1}-2\Phi_{i}+\Phi_{i-1})
$$
で決まる.ここで,$\Phi_{i}:=\phi(x_i)$及び$\Psi_{i}:=\psi(x_i)$である.なお,境界条件は$u^{n}_{0}=u^{n}_{N}=0$で決まる.

\subsection{2次元の問題}
正方領域上の波動方程式に対するDirichlet境界値問題(固定境界値問題)を数値計算することを考える.
\begin{equation}
\label{e:wave2d}
	\begin{cases}
		\partial_{tt}u(t,x,y)-c^{2}\Delta u(t,x,y)=0\quad&\text{for $(t,x,y)\in(0,\infty)\times(0,1)\times(0,1)$,}\\
		u(0,x,y)=\phi(x,y),\quad \partial_{t}u(0,x,y)=\psi(x,y)\quad&\text{for $(x,y)\in[0,1]\times[0,1]$,}\\
		u(t,x,0)=u(t,0,y)=0\quad&\text{for $(t,x,y)\in[0,\infty)\times[0,1]\times[0,1]$.}
	\end{cases}
\end{equation}

\subsection{陽解法}


内部の方程式は
$$
\frac{u^{n+1}_{i,j}-2u^{n}_{i,j}+u^{n-1}_{i,j}}{(h_{t})^{2}}-c^{2}\left(\frac{u^{n}_{i+1,j}-2u^{n}_{i,j}+u^{n}_{i-1,j}}{(h_x)^{2}}+\frac{u^{n}_{i,j+1}-2u^{n}_{i,j}+u^{n}_{i,j-1}}{(h_y)^{2}}\right)=0
$$
すなわち,
$$
f
$$




\section{いくつかの応用}
\subsection{熱移流方程式}
\subsection{分散波動方程式}





%%%%%%%%%%%%%%%%%%%%%%%%%%%%%%%%%%%%
\begin{thebibliography}{99}
  \bibitem{evans} L.~C.~Evans, \emph{Partial Differential Equations}, Partial Differential Equations, in the series Graduate studies in mathematics,
Second Edition v. 19, American Math. Society, 2010.
  \end{thebibliography}