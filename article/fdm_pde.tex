\chapter{偏微分方程式の差分法}
%% abstract %%
\begin{abstract}
もっとも基礎的な数値スキームである有限差分法(Finite Difference Method; FDM)を用いて,いくつかの有名な線形偏微分方程式を計算する.
\end{abstract}


%% Poisson %%
\section{Poisson方程式}
線形のPoisson方程式は
\begin{equation}
\label{e:poisson}
-\Delta u(x)=f(x)\quad(x\in\Omega)
\end{equation}
と記述される.

%
%\subsection{Helmholtz方程式}
%Helmholtz方程式
%\begin{equation}
%\label{e:helmholtz}
%-\Delta u(x)=\lambda u(x)\quad(x\in\Omega)
%\end{equation}



%% heat %%
\section{熱方程式}
線形の熱方程式は
\begin{equation}
\label{e:heat}
\partial_t u(t,x)-c\Delta u(t,x)=f(t,x)\quad(t>0,\ x\in\Omega)
\end{equation}
と記述される.ここで,$\Omega$は$\mathbb{R}^d$の単連結な領域であり,$c>0$は拡散係数(または粘性係数)と呼ばれる(今回は)定数である.


\subsection{ 1次元の問題}
ここでは,$\Omega=(0,1)$とし
\begin{equation}
u(0,x)=g(x)\quad(0\le x\le1)
\end{equation}
及びDirichlet境界条件
\begin{equation}
u(t,0)=a(t),\quad u(t,1)=b(t)\quad(t\ge0)
\end{equation}
を仮定する.
ここで,$g:[0,1]\to\mathbb{R}$及び$a,b:[0,\infty)\to\mathbb{R}$は与えられた関数で,両立条件として$g(0)=a(0)$と$g(1)=b(0)$を満たすと仮定する.

\subsubsection{陽解法}
時間分割数を$N_t$で空間分割数を$N_x$とする.

内部の方程式を,空間に関して\uwave{2階中心差分},時間に関して\uwave{前進差分}による近似を行うと
\begin{equation}
\label{e:heat1d}
\frac{u^{n+1}_{i}-u^{n}_{i}}{h_t}-c\frac{u^{n}_{i+1}-2u^{n}_{i}+u^{n}_{i-1}}{h_x^{2}}=f_i^n
\end{equation}
を得る.ただし,$u_i^n:=u(nh_t,ih_x)$かつ$f_i^n:=f(nh_t,ih_x)$としている.(離散)方程式\eqref{e:heat1d}は
$$
u^{n+1}_{i}=\lambda u^{n}_{i+1}+(1-2\lambda)u^{n}_{i}+\lambda u^{n}_{i-1}
$$
と書き換えられる.ここで,$\lambda:=ch_t/h_x^{2}$である.


\subsubsection{陰解法}


\subsection{2次元の問題}



\subsection{問題}
すぐに思い浮かぶ問題としては例えば次がある.
\begin{itemize}
\item Neumann境界条件を課した場合はどうなるか.
\item 拡散係数が定数でない場合,すなわち
$$
u_t(t,x)-\divergence(c(t,x)\nabla u(t,x))=f(t,x)\quad(t>0,\ x\in\Omega)
$$
の場合はどうなるか.
\end{itemize}







%% wave %%
\section{波動方程式}
線形の波動方程式は
\begin{equation}
\label{e:wave}
u_{tt}u(t,x)-c^2\Delta u(t,x)=f(t,x)\quad(t>0,\ x\in\Omega)
\end{equation}
と記述される.


\subsection{ 1次元の問題}
初期条件として
\begin{equation}
\label{e:waveinitial}
u(0,x)=g_1(x),\quad u_t(0,x)=g_2(x)\quad(0\le x\le 1)
\end{equation}
及びDirichlet境界条件
\begin{equation}
\label{e:wave1ddirichlet}
u(t,0)=a(t),\quad u(t,1)=b(t)\quad(t\ge0)
\end{equation}
を仮定する.

\subsubsection{陽解法}
内部の方程式は,
$$
\frac{u^{n+1}_{i}-2u^{n}_{i}+u^{n-1}_{i}}{(\delta t)^{2}}-c^{2}\frac{u^{n}_{i+1}-2u^{n}_{i}+u^{n}_{i-1}}{(\delta x)^{2}}=f
$$
すなわち,
$$
u^{n+1}_{i}=\lambda u^{n}_{i+1}+2(1-\lambda)u^{n}_{i}+\lambda u^{n}_{i-1}-u^{n-1}_{i}
$$
と分解される.ここで,$\lambda:=(c\delta t/\delta x)^{2}$.

初期条件は,$u^{0}_{i}=\Phi_{i}$及び
$$
u(\delta t,x)\simeq u(0,x)+\delta t\partial_{t}u(0,x)+\frac{(\delta t)^{2}}{2}\partial_{tt}u(0,x)
$$
より
$$
u^{1}_{i}=\Phi_{i}+\Psi_{i}\delta t+\frac{\lambda}{2}(\Phi_{i+1}-2\Phi_{i}+\Phi_{i-1})
$$
で決まる.ここで,$\Phi_{i}:=\phi(x_i)$及び$\Psi_{i}:=\psi(x_i)$である.なお,境界条件は$u^{n}_{0}=u^{n}_{N}=0$で決まる.

\subsection{2次元の問題}
正方領域$\Omega=(0,1)\times(0,1)$上の波動方程式\eqref{e:wave}-\eqref{e:waveinitial}に対するDirichlet境界値問題(固定境界値問題)
\begin{equation}
\label{e:wavedirichlet}
u(t,x)=a(t,x)\quad(t\ge0,\ x\in\partial\Omega)
\end{equation}
を考える.


\subsubsection{陽解法}


内部の方程式は
$$
\frac{u^{n+1}_{i,j}-2u^{n}_{i,j}+u^{n-1}_{i,j}}{(h_{t})^{2}}-c^{2}\left(\frac{u^{n}_{i+1,j}-2u^{n}_{i,j}+u^{n}_{i-1,j}}{(h_x)^{2}}+\frac{u^{n}_{i,j+1}-2u^{n}_{i,j}+u^{n}_{i,j-1}}{(h_y)^{2}}\right)=0
$$
すなわち,
$$
f
$$

\subsection{問題}
すぐに思い浮かぶ問題としては例えば次がある.
\begin{itemize}
\item Neumann境界条件を課した場合はどうなるか.
\item 拡散係数が定数でない場合,すなわち
$$
u_{tt}(t,x)-\divergence(c(t,x)\nabla u(t,x))=f(t,x)\quad(t>0,\ x\in\Omega)
$$
の場合はどうなるか.
\end{itemize}


%% advection %%
\section{移流方程式}
線形の(スカラー)移流方程式\footnote{advection equation}は
\begin{equation}
\label{e:advection}
\partial_t u(t,x)+\nu\cdot\nabla u(t,x)=f(t,x)\quad(t>0,\ x\in\Omega)
\end{equation}
によって記述される.
ここで,$u:\Omega\to\mathbb{R}$は未知関数,$\nu\in\mathbb{R}^d$はゼロでない定数ベクトルであり,$f=f(t,x):(0,\infty)\times\Omega\to\mathbb{R}$は既知の関数ある.もし$\Omega=\mathbb{R}^d$ならば,解は
$$
u(t,x)=u(0,x-t\nu)+\int_0^t f(s,x+(s-t)\nu)ds\quad(x\in\mathbb{R}^d,t\ge0)
$$
とかける(\cite[Section 2.1.2]{evans})

\subsection{離散化}




\subsection{移流拡散方程式}
前々節と合わせることで,線形の移流拡散方程式\footnote{advection diffusion equation}
\begin{equation}
\label{e:advectiondiffusion}
u_t(t,x)-c\Delta u(t,x)+\nu\cdot\nabla u(t,x)=f(t,x)\quad(t>0,\ x\in\Omega)
\end{equation}
の計算は可能である.

\subsection{問題}
すぐに思い浮かぶ問題としては例えば次がある.
\begin{itemize}
\item Neumann境界条件を課した場合はどうなるか.
\item \eqref{e:advection}において移流係数が定数でない場合\footnote{Liouville方程式と呼ばれることがあるようだ.},すなわち
$$
u_t(t,x)+\nabla(\nu(t,x) u(t,x))=f(t,x)\quad(t>0,\ x\in\Omega)
$$
の場合はどうなるか.ただし,$\nu$は既知のスカラー値関数である.
\item \eqref{e:advectiondiffusion}において移流係数または/及び拡散係数が定数でない場合\footnote{Fokker-Plank方程式と呼ばれる.},すなわち
$$
u_t(t,x)-\divergence(c(t,x)\nabla u(t,x))+\nabla(\nu(t,x)u(t,x))=f(t,x)\quad(t>0,\ x\in\Omega)
$$
の場合はどうなるか.ただし,$\nu$は既知のスカラー値関数である.
\end{itemize}




%%%%%%%%%%%%%%%%%%%%%%%%%%%%%%%%%%%%
\begin{thebibliography}{99}
  \bibitem{evans} L.~C.~Evans, \emph{Partial Differential Equations}, Partial Differential Equations, in the series Graduate studies in mathematics,
Second Edition v. 19, American Math. Society, 2010.
  \end{thebibliography}