\chapter{偏微分方程式の差分法}
%% abstract %%
\begin{abstract}
もっとも基礎的な数値スキームである有限差分法(Finite Difference Method; FDM)を用いて,いくつかの有名な線形偏微分方程式を計算する.
\end{abstract}


%% Poisson %%
\section{Poisson方程式}
線形のPoisson方程式は
\begin{equation}
\label{e:poisson}
-\Delta u(x)=f(x)\quad(x\in\Omega)
\end{equation}
と記述される.

\subsection{1次元の問題}
まず,1次元の場合($\Omega=(0,1)$)について考えてみる.一般の区間への拡張は容易である.
\begin{figure}[h]
	\begin{center}
		\begin{picture}(300,40)
     	\put(2,25){\line(1,0){182}}
		\put(240,25){\line(1,0){60}}
		\multiput(184,25)(7,0){10}{\line(1,0){3}}
		\multiput(2,25)(30,0){7}{\circle*{4}}
		\multiput(242,25)(30,0){3}{\circle*{4}}
		\put(0,10){$0$}
		\put(300,10){$1$}
		\put(0,30){$i=0$}
		\put(30,30){$1$}
		\put(59,30){$2$}
		\put(89,30){$3$}
		\put(119,30){$4$}
		\put(146,30){$\cdots$}
		\put(230,30){$n-2$}
		\put(260,30){$n-1$}
		\put(300,30){$n$}
      \end{picture}
      \caption{空間分割のイメージ($n$等分割した場合; $h=1/n$)}
    \end{center}
  \end{figure}
  
\eqref{e:poisson}は\uwave{2階の中心差分}を通して
$$
-\frac{u_{i-1}-2u_i+u_{i+1}}{h^2}=f_i\quad(i=1,\cdots,n-1),
$$
と離散化される.
ただし,$f_i=f(x_i)$である.
また,これは
\begin{equation}
\bm{A}=\left[\begin{array}{ccccc}
	2&-1&&&\\
	-1&2&-1&&\hsymb{\bm{0}}\\
	&\ddots&\ddots&\ddots&\\
	\hsymb{\bm{0}}&&-1&2&-1\\
	&&&-1&2
	\end{array}\right],\quad
\bm{u}=\left[\begin{array}{c}
	u_1\\
	u_2\\
	\vdots\\
	u_{n-1}
	\end{array}\right],\quad
\bm{u}_b=\left[\begin{array}{c}
	-u_0\\
	0\\
	\vdots\\
	0\\
	-u_{n}
	\end{array}\right],\quad
\bm{f}=\left[\begin{array}{c}
	h^2 f_1\\
	h^2 f_2\\
	\vdots\\
	h^2 f_{n-1}
	\end{array}\right]	
\end{equation}
と定義することにより,線型方程式
\begin{equation}
\label{e:poissonlinearequation}
\bm{A}\bm{u}+\bm{u}_{b}=\bm{f}
\end{equation}
に書き換えられる.
ここで,$\bm{u}_b$は境界条件から決まる.

\subsubsection{周期境界}
周期境界条件
\begin{equation}
\label{e:1dperiodic}
u(0)=u(1)
\end{equation}
を仮定する.

\subsubsection{Dirichlet問題}
Dirichlet境界値
\begin{equation}
\label{e:1dstationarydirichlet}
u(0)=a,\quad u(1)=b\quad(a,b\in\mathbb{R})
\end{equation}
を仮定すれば,$u_0=a$かつ$u_{n}=b$である.
したがって,\eqref{e:poissonlinearequation}は
$$
\bm{A}\bm{u}=\bm{f}_D
$$
へと簡約化される.なお,
\begin{equation*}
\bm{f}_D=\left[\begin{array}{c}
	h^2 f_1+a\\
	h^2 f_2\\
	\vdots\\
	h^2 f_{n-2}\\
	h^2 f_{n-1}+b
	\end{array}\right]	
\end{equation*}
である.

\subsubsection{Neumann問題}
Neumann境界値
\begin{equation}
\label{e:1dstationaryneumann}
u'(0)=a,\quad u'(1)=b\quad(a,b\in\mathbb{R})
\end{equation}
を仮定してみる.これらを
$$
\frac{u_1-u_0}{h}=a,\quad \frac{u_{n}-u_{n-1}}{h}=b
$$
と近似すれば,$u_0=u_1-ah$と$u_{n}=u_{n-1}-bh$であるので,\eqref{e:poissonlinearequation}は
$$
\bm{A}_N\bm{u}=\bm{f}_N
$$
と簡約化される.ただし,
\begin{equation}
\bm{A}_N=\left[\begin{array}{ccccc}
	1&-1&&&\\
	-1&2&-1&&\hsymb{\bm{0}}\\
	&\ddots&\ddots&\ddots&\\
	\hsymb{\bm{0}}&&-1&2&-1\\
	&&&-1&1
	\end{array}\right],\quad
\bm{f}_N=\left[\begin{array}{c}
	h^2 f_1-ah\\
	h^2 f_2\\
	\vdots\\
	h^2 f_{n-2}\\
	h^2 f_{n-1}-bh
	\end{array}\right]
\end{equation}
である.

\subsection{2次元の問題}
次に,2次元の場合を考えてみる.領域は簡単のため,正方形\footnote{矩型(くけい)領域の厳密な定義はなんでしょうかね?}($\Omega=(0,1)\times(0,1)$)とする.
長方形などへの一般化は容易である.

\eqref{e:poisson}は,より具体的には
$$
-\frac{\partial^2u}{\partial x^2}-\frac{\partial^2u}{\partial y^2}=f(x,y)
$$
と書けている.
$x$方向と$y$方向それぞれに\uwave{2階の中心差分}を適用すれば
$$
-\frac{u_{i-1,j}-2u_{i,j}+u_{i+1,j}}{h_x^2}-\frac{u_{i,j-1}-2u_{i,j}+u_{i,j+1}}{h_y^2}=f_{i,j}
$$
と離散化できる.
%
%\subsection{Helmholtz方程式}
%Helmholtz方程式
%\begin{equation}
%\label{e:helmholtz}
%-\Delta u(x)=\lambda u(x)\quad(x\in\Omega)
%\end{equation}



%% heat %%
\section{熱方程式}
線形の熱方程式は
\begin{equation}
\label{e:heat}
\partial_t u(t,x)-c\Delta u(t,x)=f(t,x)\quad(t>0,\ x\in\Omega)
\end{equation}
と記述される.ここで,$c>0$は拡散係数(または粘性係数)と呼ばれる(今回は)定数である.


\subsection{ 1次元の問題}
1次元の場合($\Omega=(0,1)$)を考える.
\begin{figure}[h]
	\begin{center}
		\begin{picture}(260,240)
		%yokosen
		\multiput(30,10)(0,30){5}{\line(1,0){120}}
		\multiput(30,180)(0,30){3}{\line(1,0){120}}
		\multiput(200,10)(0,30){5}{\line(1,0){60}}
		\multiput(200,180)(0,30){3}{\line(1,0){60}}
		%tatesen
		\multiput(30,10)(30,0){5}{\line(0,1){120}}
		\multiput(30,180)(30,0){5}{\line(0,1){60}}
		\multiput(200,10)(30,0){3}{\line(0,1){120}}
		\multiput(200,180)(30,0){3}{\line(0,1){60}}
		%dotyokosen
		\multiput(150,10)(7,0){7}{\line(1,0){3}}
		\multiput(150,40)(7,0){7}{\line(1,0){3}}
		\multiput(150,70)(7,0){7}{\line(1,0){3}}
		\multiput(150,100)(7,0){7}{\line(1,0){3}}
		\multiput(150,130)(7,0){7}{\line(1,0){3}}
		\multiput(150,180)(7,0){7}{\line(1,0){3}}
		\multiput(150,210)(7,0){7}{\line(1,0){3}}
		\multiput(150,240)(7,0){7}{\line(1,0){3}}
		%dottatesen
		\multiput(30,130)(0,7){7}{\line(0,1){3}}
		\multiput(60,130)(0,7){7}{\line(0,1){3}}
		\multiput(90,130)(0,7){7}{\line(0,1){3}}
		\multiput(120,130)(0,7){7}{\line(0,1){3}}
		\multiput(150,130)(0,7){7}{\line(0,1){3}}
		\multiput(200,130)(0,7){7}{\line(0,1){3}}
		\multiput(230,130)(0,7){7}{\line(0,1){3}}
		\multiput(260,130)(0,7){7}{\line(0,1){3}}
		%text
		\put(25,0){$i=0$}\put(58,0){$1$}\put(88,0){$2$}\put(118,0){$3$}\put(145,0){$\cdots$}\put(190,0){$n-2$}\put(220,0){$n-1$}\put(260,0){$n$}
		\put(0,8){$k=0$}\put(18,36){$1$}\put(18,66){$2$}\put(18,96){$3$}\put(18,126){$\vdots$}\put(0,180){$m-2$}\put(0,210){$m-1$}\put(18,240){$m$}
		%dot
		\put(90,100){\color{red}{\circle*{4}}}
		\put(60,70){\color{blue}{\circle*{4}}}\put(90,70){\color{blue}{\circle*{4}}}\put(120,70){\color{blue}{\circle*{4}}}
		\thicklines
		\put(62,72){\vector(1,1){24}}\put(118,72){\vector(-1,1){24}}\put(90,72){\vector(0,1){24}}
      \end{picture}
      \caption{時空間分割のイメージ(空間方向に$n$等分割,時間方向に$m$等分割した場合; $h_x=1/n$, $h_t=t_{max}/m$)}
    \end{center}
  \end{figure}






まず,空間に関して\uwave{2階の中心差分}を適用すれば
\begin{equation}
\frac{du_i(t)}{dt}-c\frac{u_{i-1}(t)+2u_i(t)-u_{i+1}(t)}{h_x^2}=f_i(t)\quad(t>0,\ i=1,\cdots,n-1)
\end{equation}
を得る.\footnote{このような離散化を(空間方向への)半離散化と呼ぶ.対して,空間方向・時間方向ともに離散化することを全離散化と呼ぶこともある.}
さらに,\uwave{前進差分}を適用することで
\begin{align}
&\frac{u^{k+1}_{i}-u^{k}_{i}}{h_t}-c\frac{u^{k}_{i-1}-2u^{k}_{i}+u^{k}_{i+1}}{h_x^{2}}=f_i^k\nonumber\\
\Leftrightarrow&\ u_i^{k+1}=\lambda u_{i-1}^k+(1-2\lambda)u_i^k+\lambda u_{i+1}^k+h_t f_i^k\quad(i=1,\cdots,n-1,\ k=0,\cdots,m-1)\label{e:1dheat}
\end{align}
が導出される.ここで,$\lambda:=ch_t/h_x^{2}$である.もし
\begin{equation}
\bm{A}=\left[\begin{array}{ccccc}
	1-2\lambda&\lambda&&&\\
	\lambda&1-2\lambda&\lambda&&\hsymb{\bm{0}}\\
	&\ddots&\ddots&\ddots&\\
	\hsymb{\bm{0}}&&\lambda&1-2\lambda&\lambda\\
	&&&\lambda&1-2\lambda
	\end{array}\right],\quad
\bm{u}^k=\left[\begin{array}{c}
	u_1^k\\
	u_2^k\\
	\vdots\\
	u_{n-1}^k
	\end{array}\right],\quad
\bm{u}_b^k=\left[\begin{array}{c}
	\lambda u_0^k\\
	0\\
	\vdots\\
	0\\
	\lambda u_{n}^k
	\end{array}\right],\quad
\bm{f}^k=\left[\begin{array}{c}
	h_t f_1^k\\
	h_t f_2^k\\
	\vdots\\
	h_t f_{n-1}^k
	\end{array}\right]	
\end{equation}
を導入するならば,\eqref{e:1dheat}は
$$
\bm{u}^{k+1}=\bm{A}\bm{u}^k + \bm{u}_b^k+ \bm{f}^k\quad(k=0,\cdots,m-1)
$$
と書くこともできる.
初期条件が
\begin{equation}
\label{e:1dinitial}
u(0,x)=g(x)\quad(0\le x\le1)
\end{equation}
と与えられたとすれば,$u_i^0=g_i$すなわち$\bm{u}^0=\bm{g}$であるので,結局
\begin{equation}
\label{e:1ddiscreteheat}
	\begin{cases}
		\bm{u}^{k+1}=\bm{A}\bm{u}^k + \bm{u}_b^k+ \bm{f}^k\quad(k=0,\cdots,m-1)&\\
		\bm{u}^0=\bm{g}&
	\end{cases}
\end{equation}
が得られる.

\subsubsection{Cauchy-Dirichlet問題}
Dirichlet境界条件
\begin{equation}
\label{e:1ddirichlet}
u(t,0)=a(t),\quad u(t,1)=b(t)\quad(t\ge0)
\end{equation}
を仮定する.
ただし,$a,b:[0,\infty)\to\mathbb{R}$は両立条件として$g(0)=a(0)$と$g(1)=b(0)$を満たすと仮定する.
\eqref{e:1ddirichlet}から$u_0^k=a^k$と$u_{n}^k=b^k$を得るので,\eqref{e:1ddiscreteheat}は
\begin{equation}
	\begin{cases}
		\bm{u}^{k+1}=\bm{A}\bm{u}^k + \bm{f}_D^k\quad(k=0,\cdots,m-1)&\\
		\bm{u}^0=\bm{g}&
	\end{cases}
\end{equation}
となる.ここで,
\begin{equation}
\bm{f}_D^k=\left[\begin{array}{c}
	h_tf_1^k+\lambda a^k\\
	h_t f_2^k\\
	\vdots\\
	h_t f_{n-2}^k\\
	h_t f_{n-1}^k + \lambda b^k
	\end{array}\right]
\end{equation}

\subsubsection{Cauchy-Neumann問題}
Neumann境界条件
\begin{equation}
\label{e:1dneumann}
u_x(t,0)=a(t),\quad u_x(t,1)=b(t)\quad(t\ge0)
\end{equation}
の場合は,
$$
\frac{u_1^k-u_0^k}{h_x}=a^k,\quad \frac{u_{n}^k-u_{n-1}^k}{h_x}=b^k
$$
と近似すれば,$u_0^k=u_1^k-a^kh_x$と$u_{n+1}^k=u_n^k-b^kh_x$であるので,\eqref{e:1ddiscreteheat}は
\begin{equation}
	\begin{cases}
		\bm{u}^{k+1}=\bm{A}_N\bm{u}^k + \bm{f}_N^k\quad(k=0,\cdots,m-1)&\\
		\bm{u}^0=\bm{g}&
	\end{cases}
\end{equation}
となる.ここで,
\begin{equation}
\bm{A}_N=\left[\begin{array}{ccccc}
	1-\lambda&\lambda&&&\\
	\lambda&1-2\lambda&\lambda&&\hsymb{\bm{0}}\\
	&\ddots&\ddots&\ddots&\\
	\hsymb{\bm{0}}&&\lambda&1-2\lambda&\lambda\\
	&&&\lambda&1-\lambda
	\end{array}\right],\quad
\bm{f}^k_N=\left[\begin{array}{c}
	h_t f_1^k-h_t a^k\\
	h_t f_2^k\\
	\vdots\\
	h_t f_{n-1}^k\\
	h_t f_{n}^k -h_t b^k
	\end{array}\right]	
\end{equation}
である.

\subsection{2次元の問題}



\subsection{問題}
\begin{itemize}
\item 他のスキームを用いた場合はどうなるか.
\item 拡散係数が定数でない場合,すなわち
$$
u_t(t,x)-\divergence(c(t,x)\nabla u(t,x))=f(t,x)\quad(t>0,\ x\in\Omega)
$$
の場合はどうなるか.
\end{itemize}







%% wave %%
\section{波動方程式}
線形の波動方程式は
\begin{equation}
\label{e:wave}
u_{tt}(t,x)-c^2\Delta u(t,x)=f(t,x)\quad(t>0,\ x\in\Omega)
\end{equation}
と記述される.


\subsection{ 1次元の問題}
1次元($\Omega=(0,1)$)の場合について考えてみる.
空間方向に\uwave{2階の中心差分}を適用すると
$$
\frac{d^2u_i(t)}{dt^2}-c^{2}\frac{u_{i-1}(t)-2u_{i}(t)+u_{i+1}(t)}{h_x^{2}}=f_i(t)\quad(t>0,\ i=1,\cdots,n-1)
$$
と半離散化される.
さらに,時間方向にも\uwave{2階の中心差分}を適用すると
\begin{align}
&\frac{u_i^{k-1}-2u_i^k+u_i^{k+1}}{h_t^2}-c^{2}\frac{u_{i-1}^k-2u_{i}^k+u_{i+1}^k}{h_x^{2}}=f_i^k\nonumber\\
\Leftrightarrow&\ u^{k+1}_i=\lambda u^k_{i-1}+2(1-\lambda)u^k_{i}+\lambda u^k_{i+1}-u^{k-1}_i+h_t^2f_i^k\quad(i=1,\cdots,n-1,\ k=1,\cdots,m-1)\label{e:1dwave}
\end{align}
を得る.ここで,$\lambda:=ch_t^2/h_x^2$.
もし
\begin{equation}
\bm{A}=\left[\begin{array}{ccccc}
	2(1-\lambda)&\lambda&&&\\
	\lambda&2(1-\lambda)&\lambda&&\hsymb{\bm{0}}\\
	&\ddots&\ddots&\ddots&\\
	\hsymb{\bm{0}}&&\lambda&2(1-\lambda)&\lambda\\
	&&&\lambda&2(1-\lambda)
	\end{array}\right],\quad
\bm{u}^k=\left[\begin{array}{c}
	u_1^k\\
	u_2^k\\
	\vdots\\
	u_{n-1}^k
	\end{array}\right],\quad
\bm{u}_b^k=\left[\begin{array}{c}
	\lambda u_0^k\\
	0\\
	\vdots\\
	0\\
	\lambda u_{n}^k
	\end{array}\right],\quad
\bm{f}^k=\left[\begin{array}{c}
	h_t^2 f_1^k\\
	h_t^2 f_2^k\\
	\vdots\\
	h_t^2 f_{n-1}^k
	\end{array}\right]	
\end{equation}
を導入するならば,\eqref{e:1dwave}は
$$
\bm{u}^{k+1} = \bm{A}\bm{u}^k + I\bm{u}^{k-1} + \bm{u}_b^k + \bm{f}^k\quad(k=1,\cdots,m-1)
$$
と行列表現できる.ここで,$I:=\diag[1,\cdots,1]$である.
初期条件として
\begin{equation}
\label{e:1dwaveinitial}
u(0,x)=g_1(x),\quad u_t(0,x)=g_2(x)\quad(0\le x\le 1)
\end{equation}
を仮定するなら,$u_i^0=(g_1)_i$と,$(u_i^1-u_i^0)/h_t=(g_2)_i$から$u_i^1=u_i^0+h_t(g_2)_i=(g_1)_i+h_t(g_2)_i$と求まるので,結局
\begin{equation}
	\begin{cases}
		\bm{u}^{k+1} = \bm{A}\bm{u}^k + I\bm{u}^{k-1} + \bm{u}_b^k + \bm{f}^k\quad&(k=1,\cdots,m-1)\\
		\bm{u}^0=\bm{g}_1,\quad \bm{u}^1=\bm{g}_1+h_t\bm{g}_2&
	\end{cases}
\end{equation}
を解くことになる.

\subsubsection{Cauchy-Dirichlet問題}
Dirichlet境界条件\eqref{e:1ddirichlet}を仮定する.

\subsubsection{Cauchy-Neumann問題}

\subsection{2次元の問題}

\subsection{問題}
\begin{itemize}
\item 拡散係数が定数でない場合,すなわち
$$
u_{tt}(t,x)-\divergence(c(t,x)\nabla u(t,x))=f(t,x)\quad(t>0,\ x\in\Omega)
$$
の場合はどうなるか.
\end{itemize}


%% advection %%
\section{移流方程式}
線形の(スカラー)移流方程式\footnote{advection equation}は
\begin{equation}
\label{e:advection}
\partial_t u(t,x)+\nu\cdot\nabla u(t,x)=f(t,x)\quad(t>0,\ x\in\Omega)
\end{equation}
によって記述される.
ここで,$u:\Omega\to\mathbb{R}$は未知関数,$\nu\in\mathbb{R}^d$はゼロでない定数ベクトルであり,$f=f(t,x):(0,\infty)\times\Omega\to\mathbb{R}$は既知の関数ある.もし$\Omega=\mathbb{R}^d$ならば,解は
$$
u(t,x)=u(0,x-t\nu)+\int_0^t f(s,x+(s-t)\nu)ds\quad(x\in\mathbb{R}^d,t\ge0)
$$
とかける(\cite[Section 2.1.2]{evans})

\subsection{離散化}




\subsection{移流拡散方程式}
前々節と合わせることで,線形の移流拡散方程式\footnote{advection diffusion equation}
\begin{equation}
\label{e:advectiondiffusion}
u_t(t,x)-c\Delta u(t,x)+\nu\cdot\nabla u(t,x)=f(t,x)\quad(t>0,\ x\in\Omega)
\end{equation}
の計算は可能である.

\subsection{問題}
すぐに思い浮かぶ問題としては例えば次がある.
\begin{itemize}
\item Neumann境界条件を課した場合はどうなるか.
\item \eqref{e:advection}において移流係数が定数でない場合\footnote{Liouville方程式と呼ばれることがあるようだ.},すなわち
$$
u_t(t,x)+\nabla(\nu(t,x) u(t,x))=f(t,x)\quad(t>0,\ x\in\Omega)
$$
の場合はどうなるか.ただし,$\nu$は既知のスカラー値関数である.
\item \eqref{e:advectiondiffusion}において移流係数または/及び拡散係数が定数でない場合\footnote{Fokker-Plank方程式と呼ばれる.},すなわち
$$
u_t(t,x)-\divergence(c(t,x)\nabla u(t,x))+\nabla(\nu(t,x)u(t,x))=f(t,x)\quad(t>0,\ x\in\Omega)
$$
の場合はどうなるか.ただし,$\nu$は既知のスカラー値関数である.
\end{itemize}










\section{アニメーション}
計算結果の可視化にはいくつか方法があるようだが,ここではアニメーションによる可視化について説明する.
\subsection{ C/C++からgnuplotを呼び出す}
C/C++でプログラムを組んでいる場合に,ソースコード内にgnuplotを呼び出すコマンドを記述しておけば,いちいち実行とは別にgnuplotを開いて実行データのプロットを行う手間が省ける.

\subsubsection{ Linuxでの方法}
Linuxで作成中の場合,\texttt{popen}関数\footnote{外部コマンドをプログラム内で使用できる一つの関数.終わりには\texttt{pclose}関数で閉じないといけない.}でgnuplotを呼び出すことができる.
\begin{lstlisting}[caption=C++のコード内からgnuplotを呼び出し$\sin x$を描く]
#include <iostream>
#include <cstdio>
using namespace std;

int main()
{
	FILE *fp = popen("gnuplot", "w");
	if (fp == NULL){
		return 1;
	}
	fputs("set mouse\n", fp);
	fputs("plot sin(x)\n", fp);
	fflush(fp);
	cin.get();
	pclose(fp);
	return 0;
}
\end{lstlisting}

\textbf{1-3行目:}
\texttt{iostream}は読み込みや書き出しを行う関数が入っているライブラリ.C言語での\texttt{stdio.h}に対応している.また,\texttt{cstdio}はFILEポインタや\texttt{fputs}関数などが入っているライブラリである.
3行目は名前空間の宣言であるが無視して構わない.

\textbf{7行目:}
FILE型でfpというファイルを用意する.そこに,\texttt{popen}関数を用いてgnuplotと書き込む(\texttt{"w"}).記号\texttt{*}はポインタを表すが別の記事を参照せよ.

\textbf{8-17行目:}
もしfpが確保できないならそこでお終い.確保できるなら,\texttt{fputs}関数で( )内の文字列をfpに書き出す.
一つ目はマウスで操作できるようにする宣言文で,二つ目は$\sin x$をプロットする宣言文である.
\texttt{fflush}関数によりFILEポインタfpのバッファに格納されているデータを吐き出させる.
最後に\texttt{pclose}でfpを閉じれば良い.


\subsubsection{ Windowsでの場合}
Windows (Visual C++)を使う場合は,マウスが自動的に有効になっているので宣言する必要はない.また,\texttt{popen}関数と\texttt{pclos}関数の代わりに\texttt{\_popen}関数と\texttt{\_pclose}関数を使う.
\begin{lstlisting}[caption=Visual C++のコード内からgnuplotを呼び出し$\sin x$を描く]
#include <iostream>
#include <cstdio>
using namespace std;

int main()
{
	FILE *fp = _popen("pgnuplot.exe", "w");
	if (fp == NULL){
		return 1;
	}else{
		fputs("plot sin(x)\n", fp);
		fflush(fp);
		cin.get();
		_pclose(fp);
		return 0;
	}
}
\end{lstlisting}



%%%%%%%%%%%%%%%%%%%%%%%%%%%%%%%%%%%%
\begin{thebibliography}{99}
  \bibitem{evans} L.~C.~Evans, \emph{Partial Differential Equations}, in the series Graduate studies in mathematics,
Second Edition v. 19, American Math. Society, 2010.
  \end{thebibliography}