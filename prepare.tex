\chapter{ルール及び事前準備}
\section{ルール}
短いコードならpdfに書き込み可.



\section{索引の書き方}

索引はindex\index{インデックス}で可能.

\section{ C/C++からgnuplotを呼び出す}
C/C++でプログラムを組んでいる場合に,ソースコード内にgnuplotを呼び出すコマンドを記述しておけば,いちいち実行とは別にgnuplotを開いて実行データのプロットを行う手間が省ける.

\subsection{ Linuxでの方法}
Linuxで作成中の場合,\texttt{popen}関数\footnote{外部コマンドをプログラム内で使用できる一つの関数.終わりには\texttt{pclose}関数で閉じないといけない.}でgnuplotを呼び出すことができる.
\begin{lstlisting}[caption=C++のコード内からgnuplotを呼び出し$\sin x$を描く]
#include <iostream>
#include <cstdio>
using namespace std;

int main()
{
	FILE *fp = popen("gnuplot", "w");
	if (fp == NULL){
		return 1;
	}
	fputs("set mouse\n", fp);
	fputs("plot sin(x)\n", fp);
	fflush(fp);
	cin.get();
	pclose(fp);
	return 0;
}
\end{lstlisting}

\subsubsection{ 説明}
\textbf{1-3行目:}
\texttt{iostream}は読み込みや書き出しを行う関数が入っているライブラリ.C言語での\texttt{stdio.h}に対応している.また,\texttt{cstdio}はFILEポインタや\texttt{fputs}関数などが入っているライブラリである.
3行目は名前空間の宣言であるが無視して構わない.

\textbf{7行目:}
FILE型でfpというファイルを用意する.そこに,\texttt{popen}関数を用いてgnuplotと書き込む(\texttt{"w"}).記号\texttt{*}はポインタを表すが別の記事を参照せよ.

\textbf{8-17行目:}
もしfpが確保できないならそこでお終い.確保できるなら,\texttt{fputs}関数で( )内の文字列をfpに書き出す.
一つ目はマウスで操作できるようにする宣言文で,二つ目は$\sin x$をプロットする宣言文である.
\texttt{fflush}関数によりFILEポインタfpのバッファに格納されているデータを吐き出させる.
最後に\texttt{pclose}でfpを閉じれば良い.


\subsection{ Windowsでの場合}
Windows (Visual C++)を使う場合は,マウスが自動的に有効になっているので宣言する必要はない.また,\texttt{popen}関数と\texttt{pclos}関数の代わりに\texttt{\_popen}関数と\texttt{\_pclose}関数を使う.
\begin{lstlisting}[caption=Visual C++のコード内からgnuplotを呼び出し$\sin x$を描く]
#include <iostream>
#include <cstdio>
using namespace std;

int main()
{
	FILE *fp = _popen("pgnuplot.exe", "w");
	if (fp == NULL){
		return 1;
	}else{
		fputs("plot sin(x)\n", fp);
		fflush(fp);
		cin.get();
		_pclose(fp);
		return 0;
	}
}
\end{lstlisting}
