\chapter{偏微分方程式の差分法}
\begin{abstract}

\end{abstract}

\section{移流方程式}
線形移流方程式
\begin{equation}
\partial_t u+\nu\cdot\nabla u=f\quad\text{in $(0,\infty)\times\Omega$}
\end{equation}
を有限差分法\footnote{finite difference method; FDM}で計算することを考える.
ここで,$\nu\in\mathbb{R}^d$はゼロでない定数ベクトルである.

\subsection{解の存在など}

\subsection{ソースコード}


\section{Poisson方程式}





\section{熱方程式}
以下の記号を用いる.
\begin{itemize}
\item[-] $\delta \ast$ ($\ast=t,x,y$):時間及び空間刻み幅
\item[-] $(t_n ,x_i ,y_j):=(n \delta t,i \delta x,j \delta y)$, ($n,i,j=0,1,2,\cdots$):時空間の格子点
\item[-] $u^{n}_{i,j}:=u(t_n, x_i, y_j)$:時空間の格子点上での未知数の値
\end{itemize}

\subsection{ 1次元の問題}
次の熱方程式に対するDirichlet境界値問題を数値計算することを考える.
\begin{equation}
\label{e:heat1d}
	\begin{cases}
		\partial_{t}u(t,x)-c\partial_{xx}u(t,x)=0\quad&\text{for $(t,x)\in(0,\infty)\times(0,1)$,}\\
		u(0,x)=\phi(x)\quad&\text{for $x\in[0,1]$,}\\
		u(t,0)=u(t,1)=0\quad&\text{for $t\in[0,\infty)$.}
	\end{cases}
\end{equation}

\subsection{陽解法}
内部の方程式は,
$$
\frac{u^{n+1}_{i}-u^{n}_{i}}{\delta t}-c\frac{u^{n}_{i+1}-2u^{n}_{i}+u^{n}_{i-1}}{(\delta x)^{2}}=0
$$
すなわち,
$$
u^{n+1}_{i}=\lambda u^{n}_{i+1}+(1-2\lambda)u^{n}_{i}+\lambda u^{n}_{i-1}
$$
と分解される.ここで,$\lambda:=c\delta t/(\delta x)^{2}$.


\subsection{ 2次元の問題}



\section{波動方程式}
以下の記号を用いる.
\begin{itemize}
\item[-] $\delta \ast$ ($\ast=t,x,y$):時間及び空間刻み幅
\item[-] $(t_n ,x_i ,y_j):=(n \delta t,i \delta x,j \delta y)$, ($n,i,j=0,1,2,\cdots$):時空間の格子点
\item[-] $u^{n}_{i,j}:=u(t_n, x_i, y_j)$:時空間の格子点上での未知数の値
\item[-]  
\end{itemize}

\subsection{ 1次元の問題}
次の波動方程式に対するDirichlet境界値問題(固定境界値問題)を数値計算することを考える.
\begin{equation}
\label{e:wave1d}
	\begin{cases}
		\partial_{tt}u(t,x)-c^{2}\partial_{xx}u(t,x)=0\quad&\text{for $(t,x)\in(0,\infty)\times(0,1)$,}\\
		u(0,x)=\phi(x),\quad \partial_{t}u(0,x)=\psi(x)\quad&\text{for $x\in[0,1]$,}\\
		u(t,0)=u(t,1)=0\quad&\text{for $t\in[0,\infty)$.}
	\end{cases}
\end{equation}

\subsubsection{陽解法}
内部の方程式は,
$$
\frac{u^{n+1}_{i}-2u^{n}_{i}+u^{n-1}_{i}}{(\delta t)^{2}}-c^{2}\frac{u^{n}_{i+1}-2u^{n}_{i}+u^{n}_{i-1}}{(\delta x)^{2}}=0
$$
すなわち,
$$
u^{n+1}_{i}=\lambda u^{n}_{i+1}+2(1-\lambda)u^{n}_{i}+\lambda u^{n}_{i-1}-u^{n-1}_{i}
$$
と分解される.ここで,$\lambda:=(c\delta t/\delta x)^{2}$.

初期条件は,$u^{0}_{i}=\Phi_{i}$及び
$$
u(\delta t,x)\simeq u(0,x)+\delta t\partial_{t}u(0,x)+\frac{(\delta t)^{2}}{2}\partial_{tt}u(0,x)
$$
より
$$
u^{1}_{i}=\Phi_{i}+\Psi_{i}\delta t+\frac{\lambda}{2}(\Phi_{i+1}-2\Phi_{i}+\Phi_{i-1})
$$
で決まる.ここで,$\Phi_{i}:=\phi(x_i)$及び$\Psi_{i}:=\psi(x_i)$である.なお,境界条件は$u^{n}_{0}=u^{n}_{N}=0$で決まる.

\begin{lstlisting}[caption=C++のコード内からgnuplotを呼び出し$\sin x$を描く]
//
//  main_wave.cpp
//  1d wave_equation: u_tt - c^2 u_xx=0 (c:const.)
//
//  Created by 難波時永 on 5/28/16.
//  Copyright (c) 2016 難波時永. All rights reserved.
//

# include <cstdlib>
# include <iostream>
# include <iomanip>
# include <fstream>
# include <ctime>
# include <cmath>
# include <cstring>

using namespace std;

# include "fd1d_wave.hpp"


void main()
{
	
}

//compute lambda
double fd1d_wave_alpha ( int x_num, double x1, double x2, int t_num, double t1, double t2, double c )
{
	double delta_t;
	double delta_x;
	double lambda;

	delta_t = ( t2 - t1 ) / ( double ) ( t_num - 1 );
	delta_x = ( x2 - x1 ) / ( double ) ( x_num - 1 );
	alpha = c * delta_t / delta_x;

	cout << "\n";
	cout << "Stability condition LAMBDA = C * DT / DX = " << lambda << "\n";

	if ( 1.0 < r8_abs ( alpha ) ){
    	cerr << "\n";
    	cerr << "FD1D_WAVE_ALPHA - Warning!\n";
    	cerr << "The stability condition |ALPHA| <= 1 fails.\n";
    	cerr << "Computed results are liable to be inaccurate.\n";
  	}

  	return alpha;
}

//first step: solve wave eq.
double *fd1d_wave_start ( int x_num, double x_vec[], double t, double t_delta, double alpha,
						double u_x1 ( double t ), double u_x2 ( double t ),
						double *ut_t1 ( int x_num, double x_vec[] ), double u1[] )
{
  	int j;
  	double *u2;
  	double *ut;

  	ut = ut_t1 ( x_num, x_vec );

  	u2 = new double[x_num];

  	u2[0] = u_x1 ( t );

  	for ( j = 1; j < x_num - 1; j++ )
  	{
    	u2[j] =           alpha * alpha   * u1[j+1] / 2.0
        	    + ( 1.0 - alpha * alpha ) * u1[j] 
            	+         alpha * alpha   * u1[j-1] / 2.0
            	+         t_delta         * ut[j];
  	}

  	u2[x_num-1] = u_x2 ( t );

  	delete [] ut;

  	return u2;
}

//computes steps
double *fd1d_wave_step ( int x_num, double t, double alpha, 
  double u_x1 ( double t ), double u_x2 ( double t ), double u1[], double u2[] )
{
	int j;
  	double *u3;

 	u3 = new double[x_num];

  	u3[0] = u_x1 ( t );

  	for ( j = 1; j < x_num - 1; j++ ){
    	u3[j] =                 alpha * alpha   * u2[j+1] 
        	  	+ 2.0 * ( 1.0 - alpha * alpha ) * u2[j]
            	+               alpha * alpha   * u2[j-1] 
            	-                                 u1[j];
  	}

  	u3[x_num-1] = u_x2 ( t );

  	return u3;
}

//evaluate a piecewise linear spline
double *piecewise_linear ( int nd, double xd[], double yd[], int nv, double xv[] )
{
  	int id;
  	int iv;
  	double *yv;

  	yv = new double[nv];

  	for ( iv = 0; iv < nv; iv++ ){
    	if ( xv[iv] < xd[0] ){
     		yv[iv] = yd[0];
    	}
    	else if ( xd[nd-1] < xv[iv] ){
   			yv[iv] = yd[nd-1];
    	}
    	else{
    		for ( id = 1; id < nd; id++ ){
        		if ( xv[iv] < xd[id] ){
            	yv[iv] = ( ( xd[id] - xv[iv]            ) * yd[id-1] 
                	   + (          xv[iv] - xd[id-1] ) * yd[id] ) 
                   	   / ( xd[id]          - xd[id-1] );
          		break;
   	        	}
        	}
    	}
  	}
	return yv;
}

//return the absolute vale of an R8
double r8_abs ( double x )
{
	double value;

  	if ( 0.0 <= x ){
 	   value = + x;
  	}
  	else{
    	value = - x;
  	}
  	return value;
}

//writes an R8MAT file
void r8mat_write ( string output_filename, int m, int n, double table[] )
{
	int i;
  	int j;
  	ofstream output;
//
//  Open the file.
//
  	output.open ( output_filename.c_str ( ) );

  	if ( !output ){
    	cerr << "\n";
    	cerr << "R8MAT_WRITE - Fatal error!\n";
    	cerr << "Could not open the output file.\n";
    	exit ( 1 );
  	}
//
//  Write the data.
//
	for ( j = 0; j < n; j++ ){
		for ( i = 0; i < m; i++ ){
	    	output << "  " << setw(24) << setprecision(16) << table[i+j*m];
    	}
    	output << "\n";
	}
//
//  Close the file.
//
  	output.close ( );

  	return;
}

//creates a vector of linearly spaced values
double *r8vec_linspace_new ( int n, double a_first, double a_last )
{
	double *a;
	int i;

  	a = new double[n];

  	if ( n == 1 ){
    	a[0] = ( a_first + a_last ) / 2.0;
    }
  	else{
    	for ( i = 0; i < n; i++ ){
		    a[i] = ( ( double ) ( n - 1 - i ) * a_first 
        	     + ( double ) (         i ) * a_last ) 
            	 / ( double ) ( n - 1     );
   		}
  	}
  	return a;
}

//prints the current YMDHMS date as a time stamp
void timestamp ( )
{
	# define TIME_SIZE 40

  	static char time_buffer[TIME_SIZE];
  	const struct std::tm *tm_ptr;
  	size_t len;
  	std::time_t now;

  	now = std::time ( NULL );
  	tm_ptr = std::localtime ( &now );

  	len = std::strftime ( time_buffer, TIME_SIZE, "%d %B %Y %I:%M:%S %p", tm_ptr );

  	std::cout << time_buffer << "\n";

  	return;
	# undef TIME_SIZE
}
\end{lstlisting}


\subsection{2次元の問題}
正方領域上の波動方程式に対するDirichlet境界値問題(固定境界値問題)を数値計算することを考える.
\begin{equation}
\label{e:wave2d}
	\begin{cases}
		\partial_{tt}u(t,x,y)-c^{2}\Delta u(t,x,y)=0\quad&\text{for $(t,x,y)\in(0,\infty)\times(0,1)\times(0,1)$,}\\
		u(0,x,y)=\phi(x,y),\quad \partial_{t}u(0,x,y)=\psi(x,y)\quad&\text{for $(x,y)\in[0,1]\times[0,1]$,}\\
		u(t,x,0)=u(t,0,y)=0\quad&\text{for $(t,x,y)\in[0,\infty)\times[0,1]\times[0,1]$.}
	\end{cases}
\end{equation}

\subsection{陽解法}


内部の方程式は
$$
\frac{u^{n+1}_{i,j}-2u^{n}_{i,j}+u^{n-1}_{i,j}}{(h_{t})^{2}}-c^{2}\left(\frac{u^{n}_{i+1,j}-2u^{n}_{i,j}+u^{n}_{i-1,j}}{(h_x)^{2}}+\frac{u^{n}_{i,j+1}-2u^{n}_{i,j}+u^{n}_{i,j-1}}{(h_y)^{2}}\right)=0
$$
すなわち,
$$
f
$$




\section{いくつかの応用}
\subsection{熱移流方程式}
\subsection{分散波動方程式}